\documentclass[letterpaper,10pt]{article}

\usepackage{CV}

%need to set this in here (this doesn't look good on regular docs, so it isn't in the sty)
\titlespacing*{\section}{0pt}{3pt}{3pt}

\setlength{\tabcolsep}{0pt}
\newcommand{\reitem}[2]{
  \begin{tabular}[t]{l}
    #2
  \end{tabular}
\hfill
\begin{tabular}[t]{r}
  #1
\end{tabular}
}


\newcommand{\sidebyside}[2]{
  %\fbox{
  \begin{minipage}[t]{.75\textwidth}
    \raggedright{}
    #2
  \end{minipage}
  %}
  %\hfill
  %\fbox{
  \hspace{.01\textwidth}
    \begin{minipage}[t]{.205\textwidth}
    \raggedleft
    #1
  \end{minipage}
%}
}

\newcommand{\quadItem}[4]{\reitem{\emph{#1}\\#2}{\textbf{#3}\\#4}}
\newcommand{\award}[3]{\textbf{#2}, #1\\#3}
\newcommand{\paper}[3]{#1. #2 #3}
\newcommand{\desc}[1]{\hspace*{.5em}#1}
\newcommand{\trio}[3]{\sidebyside{#3}{\textbf{#1}. #2}}

%***************************************************************************************************%
% Header Section %
%***************************************************************************************************%

%% Header Information
% name
\title{Publication List}
\author{andrew quinn}
% address
\def \addr {2260 Hayward Street\\Ann Arbor, MI 48105}
% phone number
\def \phone {1.630.453.1899}
% email
\def \email {\href{mailto:arquinn@umich.edu}{arquinn@umich.edu}}
\def \website {\href{https://web.eecs.umich.edu/~arquinn}{web.eecs.umich.edu/$\sim$arquinn}}


\begin{document}
\maketitle
\setlength\parindent{0pt}
\setlength{\parskip}{3pt}


%**************************************************************************************************
% Publications
%***************************************************************************************************
\section{Peer-reviewed Publications}
\begin{smenumerate}
  \item \paper{Ian Neal, \textbf{Andrew Quinn}, and Baris
    Kasikci}{\emph{Hippocrates: Healing Persistent Memory Bugs Without Doing Any
      Harm}. To appear in the Proceedings of the Twenty-Sixth International
    Conference on Architectural Support for Programming Languages and Operating
    Systems (ASPLOS). April 2021}{Acceptance Rate: $75/398=18.8\%$}
    
\item \paper{Ian Neal, Ben Reeves, Ben Stoler, \textbf{Andrew Quinn}, Youngjin Kwon,
  Simon Peter and Baris Kasikci}{\emph{Agamotto: How Persistent is your
    Persistent Memory Application?}.  In Proceedings of the 2020 USENIX
  Symposium on Operating Systems Design and Implementation (OSDI).  November
  2020.}{Acceptance Rate: $70/398=17.6\%$}

\item \paper{Matt Furlong, \textbf{Andrew Quinn}, and Jason Flinn}
      {\emph{The case for Determinism on the Edge}.  In 2nd USENIX
        Workshop on Hot Topics in Edge Computing (HotEdge).  July
        2019}{Acceptance Rate: $22/39=56\%$}

\item \paper{\textbf{Andrew Quinn}, Michael Cafarella, and Jason
  Flinn}{\emph{You can't debug what you can't see: Expanding
    observability with the OmniTable}. In Proceedings of the Workshop
  on Hot Topics in Operating Systems (HotOS).  May 2019}{Acceptance
  Rate: $30/125=24\%$}

\item \paper{\textbf{Andrew Quinn}, Jason Flinn, and Michael Cafarella}{
  \emph{Sledgehammer: Cluster-fueled Debugging}.  In Proceedings of the 2018
  USENIX Symposium on Operating Systems Design and Implementation (OSDI).
  October 2018.}{Acceptance Rate: $47/264 = 17.8\%$}

\item \paper{\textbf{Andrew Quinn}, David Devecsery, Peter M. Chen and Jason
  Flinn}{\emph{JetStream: Cluster-scale Parallelization of Information
  Flow Queries}.  In Proceedings of the 2016 USENIX Symposium on
    Operating Systems Design and Implementation (OSDI). November
  2016.}{Acceptance Rate: $47/267=17.6\%$}
\end{smenumerate}

\end{document}

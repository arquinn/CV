%%%%%%%%%%%%%%%%%%%%%%%%%%%%%%%%%%%%%%%%%%%%%%%%%%%%%%%%%%%%%%%%%%%%%%%%%%%%%%%%%%%%%%%%%%%%%%%%%%%%%

\documentclass[letterpaper,10pt]{article}

%***************************************************************************************************%
% Variables Section %
%***************************************************************************************************%

%***************************************************************************************************%
% Header Section %
%***************************************************************************************************%
% define packages used
\usepackage{CV}

%need to set this in here (this doesn't look good on regular docs, so it isn't in the sty)
\titlespacing*{\section}{0pt}{3pt}{3pt}

\setlength{\tabcolsep}{0pt}
\newcommand{\reitem}[2]{
  \begin{tabular}[t]{l}
    #2
  \end{tabular}
\hfill
\begin{tabular}[t]{r}
  #1
\end{tabular}
}


\newcommand{\sidebyside}[2]{
  %\fbox{
  \begin{minipage}[t]{.75\textwidth}
    \raggedright{}
    #2
  \end{minipage}
  %}
  %\hfill
  %\fbox{
  \hspace{.01\textwidth}
    \begin{minipage}[t]{.205\textwidth}
    \raggedleft
    #1
  \end{minipage}
%}
}

\newcommand{\quadItem}[4]{\reitem{\emph{#1}\\#2}{\textbf{#3}\\#4}}
\newcommand{\award}[3]{\textbf{#2}, #1\\#3}
\newcommand{\paper}[3]{#1. #2 #3}
\newcommand{\desc}[1]{\hspace*{.5em}#1}
\newcommand{\trio}[3]{\sidebyside{#3}{\textbf{#1}. #2}}

%***************************************************************************************************%
% Header Section %
%***************************************************************************************************%

%% Header Information
% name
\title{curriculum vitae}
\author{andrew quinn}
% address
\def \addr {2260 Hayward Street\\Ann Arbor, MI 48105}
% phone number
\def \phone {1.630.453.1899}
% email
\def \email {\href{mailto:arquinn@umich.edu}{arquinn@umich.edu}}
\def \website {\href{https://web.eecs.umich.edu/~arquinn}{web.eecs.umich.edu/$\sim$arquinn}}


\begin{document}
\maketitle
\setlength\parindent{0pt}
\setlength{\parskip}{3pt}


\begin{tabular}[t]{l}
  \addr
\end{tabular}
\hfill
\begin{tabular}[t]{r}
  Phone: \phone\\
  Email: \email\\
  Web: \website
\end{tabular}

%\vspace{0.1in}}

%***************************************************************************************************%
% Introduction Section %
%***************************************************************************************************%
\section{Introduction}

I am a Microsoft Research Fellow and a National Science Foundation Graduate
Research Fellow.

My dissertation applies techniques from data-intensive computing (i.e.,
`big-data'), such as large-scale parallelization and relational query models, to
software reliability tasks like debugging, security forensics, and data
provenance.  I have also worked on projects to improve the reliability of
applications that use emerging hardware, including persistent memory and edge
computing.

%***************************************************************************************************%
% Education Section %
%***************************************************************************************************%
\section{Education}

\quadItem{Ann Arbor, MI}{(Sep 2015--May 2021 (expected))}{University of
  Michigan}{Ph.D., Computer Science and Engineering}

\quadItem{Ann Arbor, MI}{(Sep 2015--May 2015)}{University of Michigan}{M.S.,
  Computer Science and Engineering}

\quadItem{Granville, OH}{(Aug 2010--May 2014)}{Denison University}{B.S.,
  Computer Science and B.A., Mathematics\\ Honors: \textit{summa cum
    laude}, Phi Beta Kappa, Deans List}


%***************************************************************************************************%
% Honors Section %
%***************************************************************************************************%
\section{Fellowships and Awards}
\begin{smenumerate}
\item \trio{Microsoft Research Fellowship}{Two year fellowship awarded to
  nine early-career Ph.D. students}{(2017)}

\item \trio{National Science Foundation Graduate Student Research
  Fellowship}{Three year fellowship awarded to early-career
  Ph.D. students}{(2017)}

\item \trio{John L. Gilpatrick Mathematics Award, Denison University}{Awarded to
  the most outstanding senior major in the Math and CS department}{(2014)}

\item \trio{Ted Barclay Top Five Student Athlete, Denison University}{Awarded to
  the top five student athletes at Denison University based on GPA.}{(2014)}
\end{smenumerate}
%**************************************************************************************************
% Publications
%***************************************************************************************************
\section{Peer-reviewed Publications}
\begin{smenumerate}
  \item \paper{Ian Neal, \textbf{Andrew Quinn}, and Baris
    Kasikci}{\emph{Hippocrates: Healing Persistent Memory Bugs Without Doing Any
      Harm}. To appear in the Proceedings of the Twenty-Sixth International
    Conference on Architectural Support for Programming Languages and Operating
    Systems (ASPLOS). April 2021}{Acceptance Rate: $75/398=18.8\%$}
    
\item \paper{Ian Neal, Ben Reeves, Ben Stoler, \textbf{Andrew Quinn}, Youngjin Kwon,
  Simon Peter and Baris Kasikci}{\emph{Agamotto: How Persistent is your
    Persistent Memory Application?}.  In Proceedings of the 2020 USENIX
  Symposium on Operating Systems Design and Implementation (OSDI).  November
  2020.}{Acceptance Rate: $70/398=17.6\%$}

\item \paper{Matt Furlong, \textbf{Andrew Quinn}, and Jason Flinn}
      {\emph{The case for Determinism on the Edge}.  In 2nd USENIX
        Workshop on Hot Topics in Edge Computing (HotEdge).  July
        2019}{Acceptance Rate: $22/39=56\%$}

\item \paper{\textbf{Andrew Quinn}, Michael Cafarella, and Jason
  Flinn}{\emph{You can't debug what you can't see: Expanding
    observability with the OmniTable}. In Proceedings of the Workshop
  on Hot Topics in Operating Systems (HotOS).  May 2019}{Acceptance
  Rate: $30/125=24\%$}

\item \paper{\textbf{Andrew Quinn}, Jason Flinn, and Michael Cafarella}{
  \emph{Sledgehammer: Cluster-fueled Debugging}.  In Proceedings of the 2018
  USENIX Symposium on Operating Systems Design and Implementation (OSDI).
  October 2018.}{Acceptance Rate: $47/264 = 17.8\%$}

\item \paper{\textbf{Andrew Quinn}, David Devecsery, Peter M. Chen and Jason
  Flinn}{\emph{JetStream: Cluster-scale Parallelization of Information
  Flow Queries}.  In Proceedings of the 2016 USENIX Symposium on
    Operating Systems Design and Implementation (OSDI). November
  2016.}{Acceptance Rate: $47/267=17.6\%$}
\end{smenumerate}

%**************************************************************************************************
% Presentations
%***************************************************************************************************
\section{Presentations}
\begin{smenumerate}
\item\trio{You can't debug what you can't see: Expanding Observability with the
  OmniTable}{Workshop on Hot Topics in Operating Systems (HotOS)}{(May 2019)}

\item\trio{Sledgehammer: Cluster-fueled Debugging}{USENIX Symposium on Operating
  Systems Design and Implementation (OSDI)}{(Oct 2018)}

\item\trio{JetStream: Cluster-Scale Parallelization of Information Flow
  Queries}{USENIX Symposium on Operating Systems Design and Implementation
  (OSDI)}{(Nov 2016)}

\item\trio{Power Management for Malleable Job Scheduling}{Denison University
  department of Math and Computer Science FaST talk}{(Apr 2014)}
\end{smenumerate}

%**************************************************************************************************
% Teaching
%***************************************************************************************************
\section{Teaching Experience}
\begin{smenumerate}
\item\trio{Graduate Student Instructor}{Data Structures and Algorithms,
  University of Michigan, Ann Arbor, MI}{(May 2019--Jun 2019)}

\item\trio{Computer Science Drop-in Tutor}{Introduction to Computer Science,
  Denison University, Granville, OH}{(Jan 2012--May 2014)}
\end{smenumerate}

%**************************************************************************************************
% Professional Service
%***************************************************************************************************
\section{Professional Service}
\begin{smenumerate}
\item\trio{External Review Committee Member}{Architectural Support for
  Programming Languages and Operating Systems (ASPLOS)}{(2021)}

\item\trio{Shadow Program Committee Member}{European Conference on Computer
  Systems (EuroSys)}{(2021)}

\item\trio{Ph.D. Admissions Committee Member}{University of Michigan Department
  of Computer Science and Engineering}{(2018)}
\end{smenumerate}


%**************************************************************************************************
% Experience
%***************************************************************************************************
\section{Professional Experience}
\begin{smenumerate}
\item\trio{Graduate Student}{Advisor: Dr.\ Jason Flinn and Dr.\ Peter Chen,
  University of Michigan, Ann Arbor, MI}{(Sep 2015--Present)}

\item\trio{Research Intern in the Systems Group}{Mentor: Dr.\ Suman Nath,
  Microsoft Corp., Redmond, WA}{(May 2017--Aug 2017)}

\item\trio{Software Development Engineer}{International Business Machines (IBM),
  Dublin, OH}{(Jun 2014--Aug 2015)}

\item\trio{Research Assistant in Online Algorithms}{Mentor: Dr.\ Jessen Havill,
  Denison University, Granville, OH}{(Aug 2013--May 2014)\\(May 2012--Aug 2012)}

\item\trio{Software Development Engineering Intern}{International Business
  Machines (IBM), Dublin, OH}{(May 2013--Aug 2013)}

\end{smenumerate}

%***************************************************************************************************
% Outreach
%***************************************************************************************************%
\section{Outreach Activities}
\begin{smenumerate}
\item\trio{Discover Engineering}{University of Michigan, Ann Arbor, MI}{(Aug 2019)}
\item\trio{Techie Club}{Georgian Heights Elementary, Columbus, OH}{(Aug 2014--Aug 2015)}
\item\trio{A Call to College}{Newark Elementary Schools, Newark, OH}{(Sep 2012--Dec 2013)}
\end{smenumerate}
\end{document}

